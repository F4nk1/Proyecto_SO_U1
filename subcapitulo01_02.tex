\section{Tipos de sistemas operativos}

Los sistemas operativos pueden clasificarse de diversas maneras, por la administración de usuarios, administración de tareas, manejo de recursos, entorno de destino, entre otros criterios. En esta sección se abordarán específicamente las siguientes clasificaciones: sistemas operativos monousuario y multiusuario, monotarea y multitarea, de tiempo real (RTOS), distribuidos, móviles y embebidos. A continuación, el análisis de dichas clasificaciones.

\subsection{Monousuario vs. Multiusuario}

\subsection{Monotarea vs. Multitarea}

\subsection{Tiempo real (RTOS)}

\subsection{Distribuidos}

\subsection{Móviles}

\subsection{Embebidos}

\chapter{Tipos de Sistemas Operativos}
\label{chap:os_types}

Los sistemas operativos (SO) constituyen la capa fundamental que gestiona
los recursos de hardware y provee servicios esenciales a las aplicaciones.
En la literatura académica se suelen clasificar según el número de usuarios,
el número de tareas, la naturaleza del tiempo de respuesta y el entorno en
que operan. A continuación, se analizan las principales categorías:
Monousuario vs. Multiusuario, Monotarea vs. Multitarea, Sistemas de Tiempo
Real (RTOS), Sistemas Distribuidos, Sistemas Operativos Móviles y
Embebidos.

\section{Monousuario vs. Multiusuario}

Un sistema monousuario está diseñado para atender a un solo usuario a la
vez, lo que simplifica la gestión de procesos pero limita la
interactividad concurrente. Ejemplos tempranos incluyen MS-DOS y CP/M. En
cambio, los sistemas multiusuario, como UNIX y Linux, permiten que varios
usuarios interactúen simultáneamente, compartiendo recursos de manera
segura mediante mecanismos de control de acceso y protección de memoria
\cite{silberschatz2018operating}. 

\section{Monotarea vs. Multitarea}

En sistemas monotarea, un solo proceso se ejecuta en un momento dado,
reduciendo la complejidad del planificador pero desaprovechando la
capacidad de los procesadores modernos. La multitarea introduce
algoritmos de planificación que permiten la ejecución concurrente de
procesos, ya sea mediante conmutación de contexto o multiprocesamiento
real \cite{stallings2018operating}.  

\section{Sistemas Operativos de Tiempo Real (RTOS)}

Los RTOS priorizan la predictibilidad en la respuesta, siendo esenciales
en aplicaciones críticas como automoción, aeronáutica y dispositivos
médicos. Estos sistemas distinguen entre tiempo real duro (deadlines
estrictos) y tiempo real blando (deadlines flexibles), garantizando
determinismo en la planificación \cite{buttazzo2011hard}. 

\section{Sistemas Distribuidos}

Un sistema operativo distribuido gestiona múltiples nodos como si fueran
una sola máquina lógica, proporcionando transparencia en el acceso a
recursos, tolerancia a fallos y escalabilidad. Ejemplos académicos y
comerciales incluyen Amoeba, Mach y Google Fuchsia
\cite{tanenbaum2007distributed}.  

\section{Sistemas Operativos Móviles}

Diseñados para dispositivos con limitaciones energéticas y de hardware,
los sistemas operativos móviles (Android, iOS) incorporan modelos de
seguridad basados en sandboxing y optimización de consumo energético. La
literatura destaca la evolución hacia arquitecturas híbridas con soporte
para aplicaciones en la nube \cite{enck2009understanding}.  

\section{Sistemas Operativos Embebidos}

Los sistemas embebidos operan en hardware con recursos restringidos,
integrados en productos de consumo, automóviles, telecomunicaciones y
dispositivos IoT. Estos sistemas requieren alta confiabilidad, bajo
consumo de energía y, en muchos casos, capacidades en tiempo real
\cite{marwedel2010embedded}.  

\section{Tablas y Figuras Comparativas}

En la Tabla~\ref{tab:os_comparison} se presenta una síntesis de las
características más relevantes de cada tipo de sistema operativo.

\begin{table}[H]
\centering
\begin{tabular}{|l|l|l|}
\hline
\textbf{Tipo de SO} & \textbf{Ventaja principal} & \textbf{Ejemplos} \\ \hline
Monousuario & Simplicidad & MS-DOS \\ \hline
Multiusuario & Compartición de recursos & UNIX, Linux \\ \hline
Monotarea & Bajo consumo de recursos & CP/M \\ \hline
Multitarea & Eficiencia y paralelismo & Windows, Linux \\ \hline
RTOS & Respuesta determinista & VxWorks, FreeRTOS \\ \hline
Distribuido & Escalabilidad y tolerancia a fallos & Amoeba, Mach \\ \hline
Móvil & Optimización energética y seguridad & Android, iOS \\ \hline
Embebido & Eficiencia en recursos limitados & Contiki, TinyOS \\ \hline
\end{tabular}
\caption{Comparación de tipos de sistemas operativos}
\label{tab:os_comparison}
\end{table}

\section{Discusión}

La evolución de los sistemas operativos refleja la tensión entre
eficiencia, seguridad, escalabilidad y facilidad de uso. Mientras que los
sistemas monousuario y monotarea son hoy obsoletos, su simplicidad sirvió
de base a la evolución hacia arquitecturas multitarea y multiusuario. En
la actualidad, los RTOS, embebidos y móviles ocupan un lugar central en
el ecosistema tecnológico, mientras que los distribuidos constituyen la
columna vertebral de la computación en la nube.  

