\section{SerenityOS}
\begin{figure}[H]
    \centering
    \includegraphics[width=0.3\textwidth]{figures/SerenityOS.png}
    \caption[Logo del sistema operativo SerenityOS]%
            {Logo del sistema operativo SerenityOS \citep{SerenityOS_website}}
    \label{fig:sistema_operativo_serenityos}
\end{figure}
SerenityOS es un sistema operativo desarrollado por Andreas Kling en el año 2019 basándose en un núcleo UNIX personalizado. Todo inicio a desarrollarlo desde cero en sus tiempo libre, también realizo
grabaciones en YouTube de como lo desarrollo y lo subió a GitHub para compartir todo su trabajo, para probar en maquinas virtuales como Debian y Ubuntu \citep{Ranchal2021}. Este sistema operativo esta diseñado para sistema operativo Unix simple y como una imagen para maquina virtual. 
Esta compuesto por un kernel monolítico de 32 bits con multitarea, pila de red IPv4, sistema de archivos ext2 y un gestor de composición de ventanas\citep{WikiSerenityOS_es}